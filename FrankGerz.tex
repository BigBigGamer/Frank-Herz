\input{text/diss}

\begin{document}

\def\labauthors{Виноградов И.Д., Шиков А.П.}
\def\labgroup{430}
\def\labnumber{5}
\def\labtheme{Опыт Франка-Герца}
\input{text/titlepage}

\section{Теоритическая часть}
{\bfseries Цель работы:} 
Экспериментально пронаблюдать дискретный характер поглощения энергии атомами, провести измерения потенциалов резонанса и
ионизации для атома гелия

На основании проведенных экспериментов Резерфордом в 1911 г. была построена планетарная модель атома. Но устойчивость такого атома и характер его спектров невозможно было объяснить с точки зрения известных тогда классической механики и электродинамики. Для устранения указанного противоречия Н. Бор в 1913 г. предложил квантовую теорию строения атома, в основе которой лежат следующие постулаты.


1.	Атомы могут длительно пребывать только в определенных энергетических состояниях. В этих состояниях они обладают энергиями $E_0,\,E_1,\,E_2\dots,\,E_n$, образующими дискретный ряд. При движении электронов по соответствующим этим состояниям стационарным орбитам никакого излучения или поглощения энергии не происходит.


2.	При переходе из одного энергетического состояния $E_m$, в другое $E_n$ поглощается или излучается строго определенная порция (квант) электромагнитной энергии. Энергия кванта связана с частотой излучения $\nu$ следующим отношением: $$h\nu=E_m-En,$$ где h - постоянная Планка.

Ставшие классическими эксперименты, выполненные в 1913 г. Д.Франком и Г. Герцем, непосредственно подтвердили справедливость квантовых постулатов Бора. В опыте Франка-Герца исследуются процессы столкновения электронов с атомами газа. Упрощенная схема экспериментальной установки приведена на рис.\ref{fig:1}. В баллоне лампы Д заполненной исследуемым газом, находятся три электрода: раскаленный катод К, являющийся источником электронов, сетка С и анод А. Между сеткой и катодом прикладывается разность потенциалов $\varphi_{y} =\varphi_{c} -\varphi_{k}$, ускоряющая электроны (потенциал сетки по отношению к катоду $\varphi_y$ называют ускоряющим потенциалом). Разность потенциалов между анодом и сеткой имеет, как правило, противоположный знак и носит название потенциала задержки $\varphi_{з} =\varphi_{a} -\varphi_{c}\textless{0}$.

\begin{center}
    \begin{minipage}[t]{0.49\linewidth}
        \includegraphics[width=\linewidth]{R1.png} 
        \label{fig:1}
        \vspace{-32pt}
        \captionof{figure}{} 
    \end{minipage}
    \begin{minipage}[t]{0.49\linewidth}
        \includegraphics[width=\linewidth]{1.jpg} 
        \label{fig:2}
        \vspace{-32pt}
        \captionof{figure}{} 
    \end{minipage}
\end{center}

В ходе выполнения эксперимента снимается анодно-сеточная характеристика газонаполненной лампы, т. е. зависимость анодного 
тока $i_a$ от ускоряющего потенциала $\varphi_{y}$ при постоянном потенциале задержки $\varphi_{з}$. Типичный вид этой характеристики приведен на рис.2.

На начальном участке характеристики по мере увеличения $\varphi_y$ наблюдается монотонный рост анодного тока. В этом режиме вылетающие из катода электроны при движении к сетке приобретают сравнительно малую энергию $W_e$ и сталкиваются с атомами газа упруго. При таких столкновениях кинетическая энергия атома изменяется слабо - на величину порядка $$\Delta W\sim W_e\frac{m}{M}\ll W_e,$$ где m и M - массы электрона и атома соответственно, а внутреннее состояние атома не меняется. Поскольку при столкновениях атомы отбирают у электронов лишь незначительную часть энергии, последние, проходя через некоторую эквипотенциальную поверхность с потенциалом $\varphi$, имеют энергию, примерно равную $e\varphi$ (здесь не учтена начальная скорость вылета электронов с катода).

При $\varphi_{y} \textgreater \varphi_{з}$ электроны пролетают через сетку, имея энергию, достаточную для преодоления 
задерживающего потенциала, и достигают анода. Как и в обычных электронных лампах, с ростом потенциала сетки $\varphi_{y}$ анодный
 ток возрастает. Этот процесс продолжается до тех пор, пока $\varphi_{y}$ не достигнет величины
так называемого первого критического потенциала $\varphi_{1}$ (его называют также резонансным потенциалом), при котором электроны 
приобретают энергию, достаточную для возбуждения атома. Столкновения электронов, имеющих энергию $e\varphi_{1}$, с атомами могут 
происходить неупруго. При этом электрон в процессе столкновения всю свою энергию передает атому. Величина критического
потенциала $\varphi_{1}$ связана с разностью энергии возбужденного $E_1$, и невозбужденного $E_0$ атомов законом сохранения энергии: $$e\varphi_{1}=E_1-E_0.$$

Электроны, потерявшие энергию при неупругих столкновениях, не могут преодолеть задерживающего поля между анодом и сеткой
и <<вылавливаются>> последней, поэтому анодный ток с дальнейшим ростом $\varphi_{y}$ уменьшается. Так возникает падающий участок на анодно-сеточной характеристике.

При дальнейшем увеличении $\varphi_{y}$ поверхность с потенциалом $\varphi_{1}$ (а, следовательно, и область неупругих
соударений) смещается от сетки к катоду. При $\varphi_{y}\geqslant \varphi_{1}+|\varphi_{3}|$ электроны, испытавшие
неупругие соударения на пути к сетке, вновь могут набрать энергию, превышающую $e\varphi_{з}$, и анодный ток опять
возрастает с ростом $\varphi_{y}$. Начиная со значения $\varphi_{y}\geqslant2\varphi_{1}$, электроны на своем пути могут дважды неупруго столкнуться с атомами и, потеряв энергию после второго столкновения, не преодолеть задерживающий потенциал. Это приведет к появлению второго провала на анодно-сеточной характеристике. Аналогичным образом происходит падение тока и при более высоких потенциалах $\varphi_{n}=n\varphi_{1}$.


Заметим далее, что если на длине свободного пробега электрон может набрать энергию, большую разности энергий двух уровней $E_n-E_1$, то возможно возбуждение всех уровней с
энергией, меньшей $E_n$, и даже ионизация атома, если $E_n-E_1$ больше энергии ионизации. Поэтому уменьшение длины свободного пробега $\lambda$ (за счет увеличения давления газа внутри лампы) позволяет не только увеличить точность определения резонансного потенциала, но и избежать перекрытия различных ступеней возбуждения. С другой стороны, слишком сильное уменьшение $\lambda$ нецелесообразно, т. к. при этом электроны до прихода в область неупругих соударений $\Omega$ испытывают много упругих столкновений, что увеличивает их разброс по энергиям, и, следовательно, уменьшает точность определения резонансного потенциала.

Для некоторых газов, у которых величина резонансного потенциала не сильно отличается от потенциала ионизации, можно, используя эту же лампу, только при относительно больших потенциалах задержки ($\varphi_{3} /\varphi_{y} \sim1$), измерить также и потенциал ионизации.

Для этого можно использовать то обстоятельство, что при $|\varphi_{3}|\textgreater{\varphi_{y}}$, электроны, эмитированные катодом, не достигают анода, и анодный ток может быть вызван только положительными носителями заряда. В случае $|\varphi_{3}|\textgreater{\varphi_{y}}\geqslant{\varphi_{u}}$ наличие анодного тока связано с процессами ионизации электронным ударом в окрестностях сетки. Когда $\varphi_{y}$ достигнет значения $\varphi_{u}$, у витков сетки появится область неупругих соударений, в которой энергия электронов будет достаточна для ионизации атомов газа, и возникнет ионный ток между сеткой и анодом лампы. В анодной цепи ток в этом случае будет иметь направление, противоположное обычному и для его измерения необходимо произвести <<переполюсовку>> амперметра (заметим, что название <<анод>>\,в этом случае оказывается чисто условным).

Ранее утверждалось, что для точного определения резонансного потенциала необходимо избежать перекрытия различных ступеней возбуждения; а в этом случае электроны на длине свободного пробега должны набирать энергию, не превышающую разности уровней $\text{Е}_{2}-E_{1}$. Для ионизации же необходимо, чтобы энергия, полученная электроном на длине свободного пробега, была бы не меньше $E_{u}-E_1$ ($E_{u}$ - энергия, соответствующая ионизированному атому). Казалось бы, одновременное выполнение этих двух условий невозможно. Но нельзя забывать, что картины электрических полей внутри лампы при определении резонансного потенциала и потенциала ионизации будут совершенно различными. Нетрудно убедиться, что производная $\displaystyle\dv{\varphi}{n}$ во втором случае будет существенно выше, следовательно, и электрон в этом случае может набрать на длине свободного пробега существенно большую энергию $$\Delta W_{\lambda}=e\lambda \displaystyle\dv{\varphi}{n}.$$

\newpage
\section{Экспериментальная часть}

В Данной работе в качестве рабочего газа использовался гелий, при этом давление в лампе $\rho$=1.2 мм рт.ст.
\subsection{Определение резонансного уровня}
Напряжение накала $V_H$ = 3 В \\
Напряжение задержки $V_3$= 7.5 В \\
График зависимости анодного тока от ускоряющего потенциала:

\begin{minipage}{\linewidth}
    \centering
    \includegraphics[width=\linewidth]{graphs/1.png}    
\end{minipage}

Резонансный потенциал $\varphi_1=20.5\pm 0.75$ Эв

Потенциал $\varphi_2=42.5 \pm 0.75$ Эв

Разность энергитических уровней $E_1-E_0=e\varphi_1=20.5\pm 0.75$ Эв

Табличное значение $E_1-E_0=21.2$ Эв
\subsection{Определение ионизационного потенциала}
 При разных значениях запирающего потенциала, превышающего потенциал ускорения, была снята зависимость анодного тока от
  ускоряющего потенциала. 

\begin{minipage}{\linewidth}
    \centering
    \includegraphics[width=\linewidth]{graphs/2.png}    
\end{minipage}

Потенциал ионизации $\varphi_u$ определялся как значительное увеличение анодного тока при повышении ускоряющего
потенциала. При значениях, близких, но меньше $\varphi_u$ анодный ток может появляться ввиду разброса электронов по
скоростям при эмитировании с катода. Электроны, обладающие большей начальной скоростью могут ионизировать атом, при том
что большая часть атомов останется не ионизированными. Определенный $\varphi_u$ составил $\varphi_u=23.7\ \text{Эв}$ 

Для ионизации атомов газа необходимо, чтобы энергия, которую приобретает электрон на длине свободного пробега $\Delta W_{\lambda}=e\lambda \displaystyle\dv{\varphi}{n}$
была бы не меньше энергии $E_u-E_1=e(\varphi_u-\varphi_1)=3.2\text{ Эв}.$ 


\section{Вывод}
В проведенной работе был экспериментально подтвержден вид сеточной характеристики и ионного тока. Были получены значения
$\varphi_1$, $\varphi_2$ и $\varphi_u$, напрямую связанные со свойствами исследуемого газа.
\end{document}